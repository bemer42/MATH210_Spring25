%%%%%%%%%%%%%%%%%%%%%%%%%%%%%%%%%%%%%%%%%
% Beamer Presentation
% LaTeX Template
% Version 1.0 (10/11/12)
%
% This template has been downloaded from:
% http://www.LaTeXTemplates.com
%
% License:
% CC BY-NC-SA 3.0 (http://creativecommons.org/licenses/by-nc-sa/3.0/)
%
%%%%%%%%%%%%%%%%%%%%%%%%%%%%%%%%%%%%%%%%%

%----------------------------------------------------------------------------------------
%	PACKAGES AND THEMES
%----------------------------------------------------------------------------------------

\documentclass[graphics]{beamer}

\mode<presentation> {

% The Beamer class comes with a number of default slide themes
% which change the colors and layouts of slides. Below this is a list
% of all the themes, uncomment each in turn to see what they look like.

%\usetheme{default}
%\usetheme{AnnArbor}
%\usetheme{Antibes} 
%\usetheme{Bergen}
%\usetheme{Berkeley}
%\usetheme{Berlin} **
%\usetheme{Boadilla} **
%\usetheme{CambridgeUS}
%\usetheme{Copenhagen}
%\usetheme{Darmstadt} 
%\usetheme{Dresden} 
%\usetheme{Frankfurt}
%\usetheme{Goettingen} 
%\usetheme{Hannover}
%\usetheme{Ilmenau}
\usetheme{JuanLesPins}
%\usetheme{Luebeck}
%\usetheme{Madrid}
%\usetheme{Malmoe}
%\usetheme{Marburg}
%\usetheme{Montpellier}
%\usetheme{PaloAlto}
%\usetheme{Pittsburgh}
%\usetheme{Rochester}
%\usetheme{Singapore}
%\usetheme{Szeged}
%\usetheme{Warsaw}

%\usefonttheme{professionalfonts} % using non standard fonts for beamer
%\usefonttheme{serif} % default family is serif
%\usepackage{fontspec}
%\setmainfont{Liberation Serif}

% As well as themes, the Beamer class has a number of color themes
% for any slide theme. Uncomment each of these in turn to see how it
% changes the colors of your current slide theme.

%\usecolortheme{albatross}
%\usecolortheme{beaver}
%\usecolortheme{beetle}
%\usecolortheme{crane}
%\usecolortheme{dolphin}
%\usecolortheme{dove}
%\usecolortheme{fly}
%\usecolortheme{lily}
%\usecolortheme{orchid}
%\usecolortheme{rose}
%\usecolortheme{seagull}
%\usecolortheme{seahorse}
%\usecolortheme{whale}
%\usecolortheme{wolverine}

%\setbeamertemplate{footline} % To remove the footer line in all slides uncomment this line
%\setbeamertemplate{footline}[page number] % To replace the footer line in all slides with a simple slide count uncomment this line

%\setbeamertemplate{navigation symbols}{} % To remove the navigation symbols from the bottom of all slides uncomment this line
}
\usepackage{multimedia}
\usepackage{color}
\usepackage{amssymb}
\usepackage{empheq}
\usepackage{mathrsfs}
\usepackage{enumerate}
\usepackage{pgflibraryarrows}
\usepackage{pgflibrarysnakes}
\usepackage{upgreek}
\usepackage{tipa}
\usepackage{textcomp}
\usepackage{ulem}
\usepackage{verbatim}
\usepackage{hyperref}
\usepackage{soul}
\usepackage{ulem}
\usepackage{cancel}
\usepackage{tikz}
\usefonttheme{serif}


\usepackage{graphicx} % Allows including images
\usepackage{booktabs} % Allows the use of \toprule, \midrule and \bottomrule in tables
\newcommand{\pd}[2]{\frac{\partial #1}{\partial #2}}
\newcommand{\pdd}[2]{\frac{\partial^2 #1}{\partial {#2}^2}}
\newcommand{\pddd}[2]{\frac{\partial^3 #1}{\partial {#2}^3}}
\newcommand{\de}[2]{\frac{d #1}{d #2}}
\newcommand{\wha}[1]{\widehat{\alpha_{#1}}}
\newcommand{\whk}[1]{\widehat{\kappa_{#1}}} 
\setcounter{MaxMatrixCols}{20}

%--------------------------------------------------------------------------------------------------------------------------------------
%	TITLE PAGE
%--------------------------------------------------------------------------------------------------------------------------------------
\title[Talk]{Basic Beamer Talk} 
\author[Brooks Emerick et. al. ]{Brooks Emerick, Yun Lu, and Francis Vasko \\ Kutztown University }
\institute[PIC 2022]{PIC Math Spring 2022 }
%\medskip
%\textit{bemerick@kutztown.edu}}
\date{\today }


%--------------------------------------------------------------------------------------------------------------------------------------
%	DOCUMENT
%--------------------------------------------------------------------------------------------------------------------------------------
\begin{document}

%%%%%%%%%%%%%%%%%%%%%%%%%%%%%%%%%%%%%%%%%%%%%%%%
%%%%%%%%%%%%%%%%%%%%%%%%%%%%%%%%%%%%%%%%%%%%%%%%
%%%%%%%%%%%%%%%%%%%%%%%%%%%%%%%%%%%%%%%%%%%%%%%%
%%%%%%%%%%%%%%%%%%%%%%%%%%%%%%%%%%%%%%%%%%%%%%%%
%%%%%%%%%%%%%%%%%%%%%%%%%%%%%%%%%%%%%%%%%%%%%%%%
%--------------------------------------------------------------------------------------------------------------------------------------
%	Title Slide
%--------------------------------------------------------------------------------------------------------------------------------------%%%%%%%%%%%%%%%%%%%%%%%%%%%%%%%%%%%%%%%%%%%%%%%%
%%%%%%%%%%%%%%%%%%%%%%%%%%%%%%%%%%%%%%%%%%%%%%%%
%%%%%%%%%%%%%%%%%%%%%%%%%%%%%%%%%%%%%%%%%%%%%%%%
%%%%%%%%%%%%%%%%%%%%%%%%%%%%%%%%%%%%%%%%%%%%%%%%

\begin{frame}
\titlepage 
\end{frame}




%%%%%%%%%%%%%%%%%%%%%%%%%%%%%%%%%%%%%%%%%%%%%%%%
%%%%%%%%%%%%%%%%%%%%%%%%%%%%%%%%%%%%%%%%%%%%%%%%
%%%%%%%%%%%%%%%%%%%%%%%%%%%%%%%%%%%%%%%%%%%%%%%%
%%%%%%%%%%%%%%%%%%%%%%%%%%%%%%%%%%%%%%%%%%%%%%%%
%--------------------------------------------------------------------------------------------------------------------------------------
%	Introduction to the Problem
%--------------------------------------------------------------------------------------------------------------------------------------
%%%%%%%%%%%%%%%%%%%%%%%%%%%%%%%%%%%%%%%%%%%%%%%%
%%%%%%%%%%%%%%%%%%%%%%%%%%%%%%%%%%%%%%%%%%%%%%%%
%%%%%%%%%%%%%%%%%%%%%%%%%%%%%%%%%%%%%%%%%%%%%%%%
%%%%%%%%%%%%%%%%%%%%%%%%%%%%%%%%%%%%%%%%%%%%%%%%

%---------------------------------------------------------------------------------------------------------------------------------------
\section{Introduction} 
\subsection{} 
\begin{frame}
\begin{center}
\Huge{Introduction}
\end{center}
\end{frame}
%---------------------------------------------------------------------------------------------------------------------------------------

%---------------------------------------------------------------------------------------------------------------------------------------
% Motivation Breakdown :
%---------------------------------------------------------------------------------------------------------------------------------------
\begin{frame}

\begin{itemize}
	\item \textsc{Motivation}: Literature on determining the alternative optima for combinatorial optimization problems, especially NP-Hard problems, is minimal (e.g.~determining multiple solutions to TSP \cite{Huang_2018}).\vfill
	\item The Minimum Cardinality Set Covering Problem (MCSCP) is a combinatorial integer programming problem with many applications (e.g.~ingot mold selection \cite{Vasko_1987}).  \vfill
	\item \textsc{Goal:} Develop a methodology for predicting the qualitative number of alternative optima for MCSCP.\vfill 
\end{itemize}

\end{frame}

%---------------------------------------------------------------------------------------------------------------------------------------
% Formulation of MCSCP:
%---------------------------------------------------------------------------------------------------------------------------------------
\begin{frame}
\frametitle{Formulation of MCSCP}
\small

Let $A = [a_{ij} ]$ be an $m\times n$ matrix, where $m < n$, and the entries of $A$ are zeros and ones.  Let $x = [x_j]$ be a bit string, then

\begin{align*}
	& \text{Minimize:}  \quad z = \displaystyle  \sum_{j = 1}^n x_j \tag{1} \\ 
	& \text{Subject to: } \notag \\ 
	& \qquad \qquad \sum_{j=1}^n a_{ij}x_j \geq 1 \quad \text{for} \, \, i = 1, 2, \ldots , m \tag{2.1} \\
	& \qquad \qquad \sum_{j=1}^n a_{ij} \geq 1 \quad \text{for} \, \, i = 1, 2, \ldots , m \tag{2.2}\\
	& \qquad \qquad \sum_{i=1}^m a_{ij} \geq 1 \quad \text{for} \, \, j = 1, 2, \ldots , n  \tag{2.3}\end{align*}

\end{frame}

\normalsize



%%%%%%%%%%%%%%%%%%%%%%%%%%%%%%%%%%%%%%%%%%%%%%%%
%%%%%%%%%%%%%%%%%%%%%%%%%%%%%%%%%%%%%%%%%%%%%%%%
%%%%%%%%%%%%%%%%%%%%%%%%%%%%%%%%%%%%%%%%%%%%%%%%
%%%%%%%%%%%%%%%%%%%%%%%%%%%%%%%%%%%%%%%%%%%%%%%%
%--------------------------------------------------------------------------------------------------------------------------------------
%	Introduction to the Problem
%--------------------------------------------------------------------------------------------------------------------------------------
%%%%%%%%%%%%%%%%%%%%%%%%%%%%%%%%%%%%%%%%%%%%%%%%
%%%%%%%%%%%%%%%%%%%%%%%%%%%%%%%%%%%%%%%%%%%%%%%%
%%%%%%%%%%%%%%%%%%%%%%%%%%%%%%%%%%%%%%%%%%%%%%%%
%%%%%%%%%%%%%%%%%%%%%%%%%%%%%%%%%%%%%%%%%%%%%%%%

%---------------------------------------------------------------------------------------------------------------------------------------
\section{Figures} 
\subsection{} 
\begin{frame}
\begin{center}
\Huge{Figures}
\end{center}
\end{frame}
%---------------------------------------------------------------------------------------------------------------------------------------

%---------------------------------------------------------------------------------------------------------------------------------------
% A Figure :
%---------------------------------------------------------------------------------------------------------------------------------------
\begin{frame}
\frametitle{This is an example of multiple clicks on the same slide}
\begin{overprint}

\onslide<1>
The figure hasn't appeared yet. 

\onslide<2>
Now the figure has appeared on the second click
\begin{center}
	\includegraphics[width = .75\textwidth]{Figures/Histogram.jpeg}
\end{center}

\end{overprint}
\end{frame}




%---------------------------------------------------------------------------------------------------------------------------------------
%	Reference Slide
%---------------------------------------------------------------------------------------------------------------------------------------
\begin{frame}
\frametitle{Selected References}
\begin{thebibliography}{99}
\tiny

\bibitem{Huang_2018}
T. Huang, Y. Gong, and J. Zhang.
\newblock {\em Seeking multiple solutions of combinatorial optimization problems: a proof of principle study.}
\newblock 2018 IEEE Symposium Series on Computational Intelligence (SSCI), Bangalore, India pp 1212-1218.
\bibitem{Vasko_1987}
F. J. Vasko, F. E. Wolf, and K. L. Stott.
\newblock {\em Optimal selection of ingot sizes via set covering.}
\newblock Opns Res, 38, pp 346-353, 1987.

\bibitem{Matlab_2020}
MATLAB.
\newblock {\em Statistics and Machine Learning Toolbox User's Guide, R2020a. }
\newblock 1 Apple Hill Dr, Natick, MA, 01760-2098, 2020.

\bibitem{Taha_2017}
H. Taha.
\newblock {\em Operations Research: An Introduction, 10th edition. }
\newblock Pearson, Boston, MA, 2017.

\end{thebibliography}
\end{frame}


\end{document} 