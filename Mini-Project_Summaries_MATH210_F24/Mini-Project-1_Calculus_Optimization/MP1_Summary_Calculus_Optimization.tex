\documentclass[article, 11pt]{article}

%%%%%%%%%%%%%%%%%%%%%%%%%%%%%%%%%%%%%%%%%%%%%%%%%%%
% Preamble:
%%%%%%%%%%%%%%%%%%%%%%%%%%%%%%%%%%%%%%%%%%%%%%%%%%%
% Typical Packages:
\usepackage{fullpage}
\usepackage{amsfonts}
\usepackage{amsmath}
\usepackage{amsthm}
\usepackage{graphicx}
\usepackage{color}
\usepackage{palatino, url, multicol}
\usepackage{enumerate}
\usepackage{ulem}
\usepackage{hyperref}
\usepackage{verbatim}

\thispagestyle{empty}

%%%%%%%%%%%%%%%%%%%%%%%%%%%%%%%%%%%%%%%%%%%%%%%%%%%%
% Start Document:
%%%%%%%%%%%%%%%%%%%%%%%%%%%%%%%%%%%%%%%%%%%%%%%%%%%%
\begin{document}

% Project Title:
\begin{center}\boxed{\LARGE{\textbf{\textsc{ MINI-PROJECT 1}}}}\\
\Huge{\textbf{\textsc{ Optimization Using Calculus }}} \\ \end{center}

\vspace{.6cm}

% Brief Summary:
\noindent \boxed{\textbf{\textsc{Brief Summary}}}

\vspace{-.35cm}
\noindent \hrulefill

We've all taken Calculus.  We all remember the section called ``Optimization.''  Optimization is the process of finding the best solution from a set of feasible solutions to a problem, typically by maximizing or minimizing a specific objective function while satisfying certain constraints. Different methods are used depending on the problem at hand.  One typical problem from calculus is based a farmer trying to maximize (minimize) the area (perimeter) of a rectangular pen.  The goal of this mini-project is to essentially streamline this basic exercise in calculus and to expand on it in several different ways.  For example, what if the farmer has a barn along one side of the rectangular region?  What if the farmer wants to form two rectangular pens side-by-side of equal area?  What if different parts of the fencing cost more?  Once we have a starting point figured out, we'll take this idea and expand on it in three dimensions.  

% Goals:
\noindent \boxed{\textbf{\textsc{Goals}}}

\vspace{-.35cm}
\noindent \hrulefill

\begin{itemize}
	\item Formulate and solve, by hand, the general ``calculus optimization'' problem for various scenarios.   
	\item Investigate how the solution changes based on the scenario given and visualize these changes graphically using Python. 
	\item Expand on the simpler models and build a Python program whose input are parameters that specify a particular scenario and outputs the designated solution.  
\end{itemize}

% Where to Start: 
\noindent \boxed{\textbf{\textsc{Where to Start}}}

\vspace{-.35cm}
\noindent \hrulefill

Consider the most simple two-dimensional rectangular problem of either maximizing area when the perimeter is given or minimizing perimeter when area is given.  Solve this problem, then start to add complexity.  What other types of rectangles or compartments can be considered?  How does the solution change based on these parameters?  Compute the solution \textit{by hand} then use build a program in Python that will output the solution based on the type of problem.  Once you have a base program formulated, increase the number of dimensions to three -- consider a rectangular prism (i.e.~a box).   \\

% Skills Needed/Gained:
\noindent \boxed{\textbf{\textsc{Skills Needed/Gained:}}}

\vspace{-.35cm}
\noindent \hrulefill

\begin{itemize}
	\item Formulating optimization problems with several parameters.  
	\item Applications of the derivative.  
	\item Basic programming in Python. 
	\item Document creation in \LaTeX. 
\end{itemize}

\noindent \hrulefill

\end{document}