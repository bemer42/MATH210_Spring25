\documentclass[11pt]{article}

%%%%%%%%%%%%%%%%%%%%%%%%%%%%%%%%%%%%%%%%%%%%%%%%%%%
% Preamble:
%%%%%%%%%%%%%%%%%%%%%%%%%%%%%%%%%%%%%%%%%%%%%%%%%%%
% Typical Packages:
\usepackage{fullpage}
\usepackage{amsfonts}
\usepackage{amsmath}
\usepackage{amsthm}
\usepackage{graphicx}
\usepackage{color}
\usepackage{amssymb}
\usepackage{empheq}
\usepackage{mathrsfs}
\usepackage{enumerate}
\usepackage{tikz}
\usepackage{upgreek}
\usepackage{tipa}
\usepackage{multicol}
\usepackage{verbatim}
\usepackage{floatrow}
\usepackage{gensymb}
\usepackage{caption}
\usepackage[T1]{fontenc}
\usepackage[font=small,labelfont=bf,tableposition=top]{caption}

\thispagestyle{empty}

%%%%%%%%%%%%%%%%%%%%%%%%%%%%%%%%%%%%%%%%%%%%%%%%%%%%
% Start Document:
%%%%%%%%%%%%%%%%%%%%%%%%%%%%%%%%%%%%%%%%%%%%%%%%%%%%
\begin{document}

% Title: 
\begin{center}
\Large{\textsc{Post-Course Survey }}\\
\end{center}

\vspace{.2cm}

% Demographics
\subsubsection*{Demographics}

\begin{enumerate}[{$\qquad 1.]$}]
\item What is your name?$\line(1,0){38}\line(1,0){195}$
\item What is your gender?$\line(1,0){38}\line(1,0){195}$
\item What is your major?$\line(1,0){38}\line(1,0){195}$
\item Circle your class standing: freshman, sophomore, junior, or senior. 
\end{enumerate}


% Proficiency
\subsubsection*{Proficiency}

\begin{enumerate}
\item[{$\qquad 5.]$}] In the list below, circle the computing or typesetting programs that you've learned or used in our course.  For every program that you circle, indicate your proficiency on a scale of 1 to 5, where 1 is very low proficiency and 5 is very high proficiency. 

\begin{minipage}{.3\textwidth}
\begin{itemize}
	\item Python 
	\item C++
	\item Java
	\item Excel Solver
	\item Maple
\end{itemize}
\end{minipage}
\begin{minipage}{.3\textwidth}
\begin{itemize}
	\item Mathematica
	\item Gurobi
	\item AMPL
	\item LINDO	
	\item CPLEX
\end{itemize}
\end{minipage}
\begin{minipage}{.3\textwidth}
\begin{itemize}
	\item Microsoft Word
	\item \LaTeX
	\item Powerpoint
	\item Beamer
	\item Zoom
\end{itemize}
\end{minipage}\vspace{.75cm}

Others (please specify): $\line(1,0){48}\line(1,0){195}$

\end{enumerate}

% Confidence
\subsubsection*{Confidence}

\begin{enumerate}


\item[{$\qquad 6.]$}] Consider the following statement: ``After taking the course, I am more confident that I can learn a new technology, such as Python or Excel Solver, to perform programming tasks to solve mathematical problems."  Circle one option below that indicates how strongly you agree/disagree with that statement:  
\begin{center} 
(1) Strongly Agree, \qquad (2) Agree, \qquad (3) Disagree, \qquad (4) Strongly Disagree. \end{center}


\item[{$\qquad 7.]$}] Consider the following statement: ``After taking the course, I am more confident in using a technology, such as Python or Excel Solver, to perform programming tasks to solve mathematical problems."  Circle one option below that indicates how strongly you agree/disagree with that statement:  
\begin{center} 
(1) Strongly Agree, \qquad (2) Agree, \qquad (3) Disagree, \qquad (4) Strongly Disagree. \end{center}


\item[{$\qquad 8.]$}] Consider the following statement: ``After taking the course, I am more confident using typesetting software, such as \LaTeX, to prepare scientific documents related to mathematics."  Circle one option below that indicates how strongly you agree/disagree with that statement:  
\begin{center} 
(1) Strongly Agree, \qquad (2) Agree, \qquad (3) Disagree, \qquad (4) Strongly Disagree. \end{center}

\item[{$\qquad 9.]$}] Consider the following statement: ``After taking the course, if I were to open a blank source file for Python, Excel Solver, or \LaTeX\, right now, I would probably be more sure how to start or what to do."  Circle one option below that indicates how strongly you agree/disagree with that statement:  
\begin{center} 
(1) Strongly Agree, \qquad (2) Agree, \qquad (3) Disagree, \qquad (4) Strongly Disagree. \end{center}


\end{enumerate}

% Attitude
\subsubsection*{Attitude}

\begin{enumerate}

\item[{$\qquad 10.]$}] Consider the following statement: ``After taking the course, I am more interested in learning new technology, such as Python or Excel Solver, to perform programming tasks to solve mathematical problems."  Circle one option below that indicates how strongly you agree/disagree with that statement:  
\begin{center} 
(1) Strongly Agree, \qquad (2) Agree, \qquad (3) Disagree, \qquad (4) Strongly Disagree. \end{center}

\item[{$\qquad 11.]$}] Consider the following statement: ``After taking the course, I believe more that learning a new technology, especially as it applies to mathematics, is an essential part to my educational experience."  Circle one option below that indicates how strongly you agree/disagree with that statement:  
\begin{center} 
(1) Strongly Agree, \qquad (2) Agree, \qquad (3) Disagree, \qquad (4) Strongly Disagree. \end{center}

\item[{$\qquad 12.]$}] Consider the following statement: ``After taking the course, I believe more that being proficient in a scientific computing language, such as Python or Excel Solver, will have a positive impact in my future career and make me more competitive in the job market."  Circle one option below that indicates how strongly you agree/disagree with that statement:  
\begin{center} 
(1) Strongly Agree, \qquad (2) Agree, \qquad (3) Disagree, \qquad (4) Strongly Disagree. \end{center}

\item[{$\qquad 13.]$}] Consider the following statement: ``After taking the course, I believe more that being proficient in a typesetting language, such as \LaTeX, is essential to becoming a better communicator in written and/or oral format, especially when conveying mathematical ideas."  Circle one option below that indicates how strongly you agree/disagree with that statement:  
\begin{center} 
(1) Strongly Agree, \qquad (2) Agree, \qquad (3) Disagree, \qquad (4) Strongly Disagree. \end{center}

\end{enumerate}


% Attitude
\subsubsection*{Overall}

\begin{enumerate}

\item[{$\qquad 14.]$}] How do you feel the projects in our class help you achieve the proficiency in the use of technology in mathematics? Any suggestions for improvement? \vfill


\end{enumerate}

\end{document}